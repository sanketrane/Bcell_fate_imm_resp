% Typeset with XeTeX
% Allows use of system fonts rather than just LaTeX's ones
% NOTE - if you use TeXShop and Bibdesk (Mac), can complete citations
%  - open your .bib file, type~\citep{xx... and then F5 or Option-Escape
\documentclass[11pt]{article}
\usepackage[margin=1in, letterpaper]{geometry} % set page layout
%\geometry{letterpaper}  % or a4paper
\usepackage[xetex]{graphicx} % allows us to manipulate graphics.
% Replace option [] with pdftex if you don't use Xe(La)TeX
\usepackage{color}
\usepackage{indentfirst}
\usepackage{hyphenat}
\usepackage{epstopdf} % automatic conversion of eps to pdf 
\usepackage{amsmath, amssymb} % Better maths support & more symbols
\usepackage{textcomp} % provide lots of new symbols - see textcomp.pdf
% line spacing: \doublespacing, \onehalfspacing, \singlespacing
\usepackage{setspace}
\usepackage[hidelinks]{hyperref}

\usepackage[table]{xcolor}% http://ctan.org/pkg/xcolor

\singlespacing
\usepackage{pgfplotstable}
% allows text flowing around figs
% use \begin{wrapfigure}{x}{width} where x = r(ight) or l(eft)
\usepackage{wrapfig}
\usepackage[parfill]{parskip} % don't indent new paragraphs
\usepackage{flafter}  % Don't place figs & tables before their definition 
\usepackage{verbatim} % allows \begin and \end{comment} regions
\usepackage{booktabs} % makes tables look good
\usepackage{bm}  % Define \bm{} to use bold math fonts
% linenumbers in L margin, start & end with \linenumbers \nolinenumbers,
\usepackage{lineno} % use option [modulo] for steps of 5
\usepackage[auth-sc]{authblk} % authors & institutions - see authblk.pdf
%\renewcommand\Authands{ and } % separates the last 2 authors in the list
% control how captions look; here, use small font and indent both margins by 20pt
\usepackage[small]{caption} 
\setlength{\captionmargin}{20pt}

%: FONT
% If you don't want to use system fonts, replace from here to 'Citation style' with \usepackage{Palatino} or similar
%: ************ FANCY FONTS START HERE
\usepackage[no-math]{fontspec} % 'no-math' = keep computer modern for math fonts
\usepackage{xunicode} % needed by XeTeX for handling all the system fonts nicely
\usepackage[no-sscript]{xltxtra} 
%\setmonofont[Scale=0.8]{PT Serif} % typeface for \tt commands
%\setsansfont[BoldFont={PT Serif Bold}, ItalicFont={PT Serif Italic}]{PT Serif} 
\defaultfontfeatures{Mapping=tex-text}
\setmainfont{Helvetica}
%\setmainfont{Source Sans Pro}

%: ************ FANCY FONTS END HERE

%:CITATION STYLE
% natbib package: square,curly, angle(brackets)
% colon (default), comma (to separate multiple citations)
% authoryear (default),numbers (citations style)
% super (for superscripted numerical citations, as in Nature)
% sort (orders multiple cites into order of appearance in ref list, or year if authoryear)
% sort&compress: as sort, + multiple citations compressed (as 3-6, 15)
\usepackage[numbers,comma,sort&compress]{natbib}

%:SHORTCUT COMMANDS
% Maths
\newcommand{\ddt}[1]{\ensuremath{\frac{{\rm d}#1}{{\rm d}t}}}  % d/dt
\newcommand{\dd}[2]{\ensuremath{\frac{{\rm d}#1}{{\rm d}#2}}} % dy by dx  - \dd{y}{x}
\newcommand{\ddsq}[2]{\ensuremath{\frac{{\rm d}^2#1}{{\rm d}#2^2}}} % second deriv
\newcommand{\pp}[2]{\ensuremath{\frac{\partial #1}{\partial #2}}} % partial \pp{y}{x}
\newcommand{\ppsq}[2]{\ensuremath{\frac{\partial^2 #1}{\partial {#2}^2}}}
\newcommand{\superscript}[1]{\ensuremath{^{\textrm{#1}}}} %normal (non-math) font for super/subscripts in text
\newcommand{\subscript}[1]{\ensuremath{_{\textrm{#1}}}}
\newcommand{\positive}{\ensuremath{^+}}
\newcommand{\negative}{\ensuremath{^-}}
% Editing
\newcommand{\gray}[1]{{\color{gray}{#1}}}
\newcommand{\red}[1]{{\color{red}{#1}}}
\newcommand{\redtext}[1]{{\color{red}{#1}}}
\newcommand{\blue}[1]{{\color{blue}{#1}}}
\newcommand{\bluetext}[1]{{\color{blue}{#1}}}
\newcommand{\green}[1]{{\color{green}{#1}}}
\newcommand{\scinot}[2]{\ensuremath{#1 \times 10^{#2}}}
% Standard stuff
\newcommand{\be}{\begin{equation}}
\newcommand{\ee}{\end{equation}}
\newcommand{\bea}{\begin{eqnarray}}
\newcommand{\eea}{\end{eqnarray}}
\newcommand{\ie}{\textit{i.e.}}
\newcommand{\etal}{\textit{et al.}}
\newcommand{\khi}{Ki67$^\text{hi}$}
\newcommand{\klo}{Ki67$^\text{lo}$}


%%%%%%%%%%%%%%%%%%%%%%%


\title{Mathematical modeling of MZ B cells during an immune response}
\author{}

\date{}

\begin{document} 
	%\maketitle
	
\textbf{Response to the comments from reviewer - 3}

\gray{
1) Are the ``$\nu$''  parameters which appear throughout (for example in supplemental equation 1 and main text equations 2 and 6 taken to be the same). If not, can they be assigned separate symbols to avoid confusion.
}

Response: Thank you for pointing this out. We have corrected this in the revised version.


\gray{
2) It was unclear how the various processes in the model should depend on the existence of Notch2 signaling. If I correctly understood the details, aside from the decay of the CAR$^+$MZB cells, these parameters seem to be taken to be the same for the “wild-type” and the knockout. It was not clear what the basis was for this assumption. why for example was the $\mu$ production rate not dependent on Notch2 signaling given the statement in the discussion that this signaling induces MZB development. In general, the model choices could have been connected better to the underlying biology.
} 

Response: We appreciate the detailed scrutiny of our models but there seems to be a minor misunderstanding.
We indeed do consider differences in generation of CAR$^{+}$ MZB cells between wild-type (control/CAR) and Notch2 knockout (N2KO//CAR) mice.
In N2KO//CAR mice, we assume complete absence of differentiation of CAR$^{+}$ (activated) FoB  or GCB cells into MZB cells, since activation induces deletion of Notch2 receptor in their B cells.(\textit{i.e.} $\mu=0$).
%In control/CAR mice $\mu \neq 0$.
Notch2 is indispensable for MZB cell generation (PMID: 11967543, 12753744, 15146182).
To clarify this point further we have modified \red{Fig. 7 (panels C and D)} and the description of mathematical models in the methods section.
%%The concern that all parameters except the decay rate of the CAR$^{+}$ MZB cells are identical between wild-type and the knockout scenarios, is a misinterpretation.
%Our startegies to model data from for wild-type (control/CAR) and Notch2 knockout (N2KO//CAR) mice differ in two ways:
%\begin{enumerate}
%\item %Differences in loss of CAR$^{+}$ MZB cells.
%We define different loss rates CAR$^{+}$ MZB cells in control/CAR and N2KO//CAR mice, $\lambda$ and $\lambda$' respectively).
%\item Differences in the generation of CAR$^{+}$ MZB cells.
%In N2KO//CAR mice, activation induces deletion of Notch2 receptor in B cells.
%We assume complete absence of differentiation of CAR$^{+}$ (activated) FoB  or GCB cells into MZB cells in these mice (\textit{i.e.} $\mu=0$), since Notch2 is shown to be indispensable for MZB cell generation (PMID: 11967543, 12753744, 15146182).
%To clarify this point further we have modified the Fig. 7 (panels C and D) and the description of mathematical models in the methods section.
%\end{enumerate} 
%We consider activation of pre-existing MZB cells ($\beta$) to be Notch2 independent and is therefore described by same rate constant ($\beta$) in both wild-type and Notch2 knockout mice. 

\blue{Depending on the tone that you are going for you may either keep or remove this: \\
We believe we are known in the field for taking great care in developing approaches that integrate theory and biology closely; and would like to argue that our model-design here too is well tethered to the underlying biology of the experimental system and suited for the questions we are asking. 
}

\gray{
3) In general, I believe that mathematical models should try to predict new features rather than just recapitulate existing data. Are there any new predictions for further experiments that the authors can propose, as this would increase the usefulness of the modeling section.
}

Response: We fully agree with this viewpoint and thank the reviewer for suggesting it. 
\blue{
%The models described here can be adopted to predict immune-response dynamics of MZB cells in diverse experimental settings.
We now include simulations in which we adopt our models to predict immune-response dynamics of MZB cells in an adoptive transfer setting. 
Specifically, we simulated a transplantation experiment in which MZB cell generation from donor FOB and GCB cells (CD45.1 and CD45.2 backgrounds, respectively) is tracked in congenic (CD45.1/CD45.2 background) recipient mice.
The virtual transplantation experiment is described in detail in supplementary text \red{XX} and schematically in the supplementary figure \red{XX}.
Using the parameters derived from our model-fitting to the CAR-expression data, we generated predictions for donor MZB cell developmental dynamics during an immune response, assuming either (i) the branched model or (ii) the linear model or (ii) a hybrid model in which both branched and linear pathways operate to govern B cell fate decisions.
This theoretical forecasting, identified critical components of the adoptive transfer experimental design and provides a means to limit and select potential outcomes once the experiment is performed. 
%Such theoretical forecasting is used routinely in systems immunology to limit and select potential experiments, to identify critical components of their design, and to provide external validation once the experiments are performed. 
}

\clearpage


\section*{Predicting B cell diversification patterns in response to immunization in an adoptive transfer strategy}
%Our fitting procedure identified branched model as the dominant pathway of CAR$^{+}$ MZB cell generation during NP-CGG response. 
%However, the true causal structure of immunization induced MZB generation may include contributions from both branched and linear models.
We simulated an experiment in which CAR$^{+}$ FoB and GCB cells are sorted on day 7 post immunization from congenic control/CAR mice of CD45.1 and CD45.2 backgrounds, respectively. 
Sorted cells are then co-transferred in equal proportions in wild-type recipient mice (CD45.1 CD45.2 double positive background) that were immunized synchronously with donor mice (Supplementary Fig. \red{XX}A). 
In this system, CAR expression identifies activated donor-derived B cells. Within CAR$^{+}$ subset, CD45.1/CD45.2 expression distinguishes MZB cells developed from either FoB or GCB cells.
We generated predictions for the dynamics of CAR$^{+}$  MZB and GCB cells, considering either (a) branched model, (b) linear model, or (c) a hybrid model in which both branched and linear pathways govern B cell dynamics during immune responses. 
We assumed 10\% grafting efficiency for $10^{6}$ FoB (CD45.1) and $10^{6}$ GCB (CD45.2) transferred cells. 

We used the mean values of parameters estimated from our analysis shown in Fig. 8, where we fitted the branched and linear models separately to the CAR expression dynamics in control/CAR mice during the TD response.
%Our models did not account for the loss of activated (CAR$^{+}$) FoB cells either by death or extra-follicular differentiation into plasma or memory fates.
Our prior analysis did not estimate the rate of loss ($\phi$, either by death or by differentiation into plasma or memory fates) of CAR$^{+}$ FoB cells.
For the purpose of this simulation, we explored four values of the loss rate of transferred FoB cells \textit{viz.} $\phi$ = 0.01, 0.5, 1 and 5 day$^{-1}$.

\textbf{Results of the simulations:}

\textbf{(a) Branched model:}
In this model, all CAR$^{+}$  MZ B cells are generated from FoB CD45.1$^{+}$ donor cells.
CD45.2$^{+}$ GCB cells do not differentiate into MZ B cells.
%Since, the linear pathway is not permitted CD45.2$^{+}$ MZB cells are not generated from differentiation of CD45.2$^{+}$ GC B cells.  
The fraction of CD45.1$^{+}$ cells in donor CAR$^{+}$ MZB cell compartment always equals 100\%~(Supplementary Fig.~\red{XX}B).

\textbf{(b) Linear model:}
In this model, CAR$^{+}$ MZB subset contains both CD45.1$^{+}$ and CD45.2$^{+}$ cells.
CAR$^{+}$ CD45.2$^{+}$ MZB cells are generated from linear differentiation of transferred CD45.2$^{+}$ GCB cells.
CAR$^{+}$ CD45.1$^{+}$ MZB cells are generated from CD45.1$^{+}$ GCB cells, which come from differentiation of transferred CD45.1$^{+}$ FoB cells.
Our simulation of the linear model predicts that immediately post transfer donor MZB compartment predominantly contains CD45.2$^{+}$ cells~(Supplementary Fig.~\red{XX}C), and their fraction declines over time as CD45.1$^{+}$ MZB cells gradually emerge from CD45.1$^{+}$ GCB cells. 
Additionally, the CD45.2 fraction in donor MZB subset approaches 1 with increasing values of the loss rate of CAR$^{+}$ FoB cells~(Supplementary Fig.~\red{XX}C). 


\textbf{(c) Hybrid (branched + linear) model:}
In this scenario, CD45.1$^{+}$ cells in CAR$^{+}$ MZB compartment are generated from CD45.1$^{+}$ FoB and CD45.1$^{+}$ GCB cells.  %that are generated from transferred FoBs. 
CD45.2$^{+}$ cells in CAR$^{+}$  MZB compartment come from linear differentiation of CD45.2$^{+}$ GCB cells.
Our simulation suggests that the dynamics of CD45.1$^{+}$ and CD45.2$^{+}$ cells in CAR$^{+}$ MZB compartment relies strongly on the loss rate of CAR$^{+}$ FoB cells --  $\phi$.
At values of $\phi <$ 0.7, the fraction of CD45.1$^{+}$ cells in the donor MZ B compartment is higher than the fraction of CD45.2$^{+}$ cells.
This pattern reverses for values of $\phi \ge$ 0.7, and the number of CD45.2$^{+}$ cells surpasses the number of  CD45.1$^{+}$ cells in the CAR$^{+}$  MZB compartment~(Supplementary Fig.~\red{XX}D).

Our simulations predict distinct quantitative and qualitative outcomes for the branched, linear and hybrid models.
Our analysis here suggests that mathematically modeling the data-derived from this co-transfer experiment can effectively untangle the contributions of branched and linear pathways to MZ B cell generation during immune responses.


%: Supplementary Fig 9 - simulation of the co-transfer experiment
\begin{figure}[h!]
\center
\includegraphics[width=0.85\textwidth]{supp-9V2.pdf}
\caption{
    \textbf{Simulation of the co-transfer experiment using parameters derived from modeling the TD response in control/CAR mice.}
    \textbf{(A)} Experimental design of the adoptive transfer strategy.
    \textbf{(B-D)} Results from the simulations of (B) branched, (C) linear and (D) branched + linear models of MZ B cell generation during an active immune response.
    Plots on the left  show cell counts of GC (GC.1), MZ (MZ.1) B cells from CD45.1 and GC (GC.2), MZ (MZ.2) B cells from CD45.2 donors.
    Plots on the right depict fraction of CD45.1 and CD45.2 donor cells in CAR$^+$ MZ B cell compartment.
    Individual panels represent predictions generated using a fixed value of the loss rate ($\phi$) of CAR$^{+}$ FoB cells.
    }
    \label{fig:co-transf}
 \end{figure}
 

\clearpage



\section*{Predicting the clonal structure of MZ B cells generated from antigen induced activation of FoB cells}

To simulate the dynamics of antigen-specific clones in the MZ B cell pool during the TD response, we reformulated our best-fit branched model into a probabilistic approach.
We considered that `$N$' clones in FoB population respond to the TD antigen and explored a simple scenario in which responding clones exhibit similar kinetics of recruitment and retention in the MZ B cell pool.
%We considered that CAR$^{+}$ FoB cell pool is  
We described the dynamics in CAR$^{+}$ FoB cells using a time-varying function $B_0 + B_1 \, t^p  (1 - \frac{t^q}{X^q + t^q})$. %Fig.
The total size of CAR$^{+}$ FoB cells is an aggregate of sizes of responding $N$ clones, at any give time.
The initial distribution of antigen-specific clones in the FoB cell compartment is defined using the power-law.
We further considered that this distribution of antigen-specific FoB cell clones varies with time, potentially due to differences in antigen accessibility and/or clonal response dynamics. 
We used the mean estimates of the \textit{per capita} rates of influx ($\mu = 0.0014$) and loss ($\lambda = 0.49$) derived from fitting the branched time-varying influx model to the data from control/CAR mice.
As an example, we simulated 75 clones considering a scenario in which clones that are highly ranked (based on their size) at the beginning of the response are enriched over time in the FoB pool~(\textbf{Fig.~\ref{fig:clone-dyn}A}).
We considered that all 75 clones can enter MZB pool and generated predictions of their clone size distributions for up to 2 months post immunization~(\textbf{Fig.~\ref{fig:clone-dyn}B}), following a probabilistic model described below.

%We considered that birth and loss of MZ B clones are poisson processes and therefore their wait times follow the exponential distribution.
%For example, the probability of a loss (by death or differentiation) event in a small time-interval $\tau$ is $1-e^{-\lambda\,\tau}$.
 
\subsection*{Stochastic simulation of clonal dynamics:} 
%Number of CAR-expressing FoB cells at time $t$ are given as, $F(t) \text{ and } M(t)$, respectively.
Consider in a small time interval $s$, the probabilities of recruitment and persistence of an antigen-specific B cell clone in the MZB cell pool are $P_\text{recruit} = 1-e^{-\mu \, s} $  and $P_\text{persist} = 1 - P_\text{loss} =  e^{-\lambda \, s} $.

The normalized density $(x)$ and the total size $(X)$ of a clone $n$ in the CAR$^{+}$ FoB pool at time $t$ are defined as,
$$
x(n, t) = \frac{n^{-k} \, (1-k)}{N^{1-k}} \quad  \text{ and }  %\quad  \text{where } n = 1 \dots N. 
\quad 
X(n,t) = F(t) \, x(n,t),
$$
where $F(t)$ is the number of total CAR$^{+}$ FoB cells at time $t$.
We define the distribution of antigen-specific clones in the CAR$^{+}$ MZB pool as $M(n,t)$, such that the total CAR$^{+}$ MZB pool size at time $t$ is $\int_{0}^{N} M(n, t) \,dn = M(t)$. We initiate this distribution at $t_0$ as $M(n, 0) = 0$ for clones $n=1, \dots, N$.
The model then takes small time-steps of size $\tau$ to update $M(n, t)$ for $t = s, 2\,s, \dots ,T_\text{max}$ by modeling influx and loss events sequentially:


At time step $t + s$, %(for $t = 0, \dots ,60$), \\

\begin{tabular}{ll}
\textbf{Influx} & Number of new clones recruited into the CAR$^{+}$ MZB pool: $y \sim \text{Binom}(F(t+s), P_\text{recruit})$. \\
& We draw a random sample of $y$ clones from $F(n, t+s)$ as $Y(n)$. \\
& The clonal distribution and the pool size of CAR$^{+}$ MZB cells are: \\ 
& $\qquad U(n, t+s) = M(n, t) + Y(n)$ for $n=1, \dots, N \; $ and \\
& $\qquad U(t+s) = \int_{0}^{N} U(n, t+s) \, dn$. \\
\\
\textbf{Loss} & Number clones that persist in the pool: $z \sim \text{Binom}(U(t+s), P_\text{persist})$.  
\end{tabular}

The updated clonal distribution $M(n, t+s)$ of the CAR$^{+}$ MZB cells at the end of time-step $t+s$ is generated by drawing a random sample of $z$ clones from $U(n, t+s)$. Their updated pool size is given as,
$M(t+s) = \int_{0}^{N} M(n, t+s) \, dn.$
\begin{figure}[!h]
\centering
\includegraphics[width=0.8\textwidth]{sim_2.pdf}
\caption{
    \textbf{Stochastic simulation of the clonal dynamics during the TD response.}
    }
\label{fig:clone-dyn} 
\end{figure}


\end{document}








